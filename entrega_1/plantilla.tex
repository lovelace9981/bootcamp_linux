\input{preambulo.tex}

%----------------------------------------------------------------------------------------
%	TÍTULO Y DATOS DEL ALUMNO
%----------------------------------------------------------------------------------------
\graphicspath{ {img/} }

\title{
\normalfont \normalsize
\includegraphics[width=6cm,height=6cm]{logo}\\
\textsc{\textbf{Bootcamp Especialidad GNU/Linux (2023)}} \\ [25pt] % Your university, school and/or department name(s)
\horrule{0.5pt} \\[0.4cm] % Thin top horizontal rule
\huge Lab 01 - Instalación de una distribución de GNU/Linux \\ % The assignment title
\horrule{2pt} \\[0.5cm] % Thick bottom horizontal rule
}

%https://es.overleaf.com/learn/latex/Inserting_Images
%Ruta relativa de   imagenes

\author{Pedro Antonio Mayorgas Parejo} % Nombre y apellidos

\date{\normalsize\today} % Incluye la fecha actual

%----------------------------------------------------------------------------------------
% DOCUMENTO
%----------------------------------------------------------------------------------------

\begin{document}

\maketitle % Muestra el Título

\newpage %inserta un salto de página

\tableofcontents % para generar el índice de contenidos

\newpage

%----------------------------------------------------------------------------------------
%	Cuestión 1
%----------------------------------------------------------------------------------------

\section{Configuración de la Máquina Virtual}

Se ha utilizado QEMU/KVM como sistema hypervisor baremetal, ya que el sistema operativo base que se utiliza, es una distribución de GNU/Linux, permitiendo que pueda tener todas ventajas de aceleración casi sin penalizaciones en el rendimiento.
\vspace{5mm}

Para la configuración de la máquina virtual, he realizado una configuración base consistente en los siguientes parámetros:

\begin{itemize}
\item 2 VCPU, donde asignamos un mínimo de 2 CPU para la fluidez y la multitarea real.
\item 4096 MiB de Memoria RAM, que aunque sea demasiado para las tareas que se le van a asignar a dicha máquina virutal, nos asegura margen futuro.
\item Disco Virtual formato QCOW2, de un tamaño de 40 GiB. 
\end{itemize}

Pasos seguidos para el proceso de instalación de la distribución en una Máquina Virtual.

\begin{figure}[H]
	\centering
	\includegraphics[scale=0.40]{00}
	\caption{La configuración comienza preguntando el medio de instalación.}
\end{figure}

\begin{figure}[H]
	\centering
	\includegraphics[scale=0.40]{01}
	\caption{Seleccionamos la ISO de Ubuntu Server desde la ruta que queramos y el instalador detecta qué sistema es.}
\end{figure}

\begin{figure}[H]
	\centering
	\includegraphics[scale=0.40]{02}
	\caption{Ahora en este punto de la instalación se configura la capacidad de la memoria RAM y CPU.}
\end{figure}

\begin{figure}[H]
	\centering
	\includegraphics[scale=0.40]{03}
	\caption{Ahora en este punto de la instalación se configura la capacidad de la memoria RAM y CPU.}
\end{figure}

\begin{figure}[H]
	\centering
	\includegraphics[scale=0.40]{04}
	\caption{Ahora configuramos el disco duro y su capacidad.}
\end{figure}

\begin{figure}[H]
	\centering
	\includegraphics[scale=0.40]{05}
	\caption{Ventana resumen de la máquina.}
\end{figure}

\newpage
\section{Configuración del sistema operativo base}

En la instalación después de seleccionar el idioma y la región, nos mostrará que versión queremos instalar de Ubuntu Server.

\begin{figure}[H]
	\centering
	\includegraphics[scale=0.40]{05}
	\caption{Configuración de la versión de Ubuntu Server.}
\end{figure}

Ahora la configuración de la interfaz de red, que es obtenida por DHCPv4 por parte del sistema hypervisor con una puerta de enlace.

\begin{figure}[H]
	\centering
	\includegraphics[scale=0.40]{06}
	\caption{Configuración de red.}
\end{figure}

Ahora comenzamos a ver la configuración de los discos, en concreto el particionado. \textbf{Seleccionamos el particionado manual}.

\begin{figure}[H]
	\centering
	\includegraphics[scale=0.40]{07}
	\caption{Selección del particionado manual.}
\end{figure}

Dentro de la herramienta de particionado manual, debemos crear una partición EFI ESP, que nos permita alojar ahí las configuraciones, variables y ficheros de bootloader (arranque). Dicha partición está formateada en \textbf{fat32}. La partición se monta en \textbf{/boot/efi}.

\begin{figure}[H]
	\centering
	\includegraphics[scale=0.30]{08}
	\caption{Creación de la partición de arranque.}
\end{figure}

Ahora creamos un VG (Volume Group) de LVM a partir de la partición 2 como PV (Physical Volume), donde a partir de dicho VG, se crea un LV (Logical Volume), que funcionará como "partición" que alojará el sistema de ficheros. 

\begin{figure}[H]
	\centering
	\includegraphics[scale=0.30]{09}
	\caption{Creación del VG a partir del PV de la partición 2.}
\end{figure}

Ahora a continuación, como se ha dicho en el párrafo anterior, tenemos que crear el LV. Para ello nos vamos a \textbf{FREE SPACE} y nos aparecerá la ventana de creación de un LV.

\begin{figure}[H]
	\centering
	\includegraphics[scale=0.40]{10}
	\caption{Creación del LV menú contextual.}
\end{figure}

Configuración para la raíz \textbf{(ROOT o /)}.

\begin{figure}[H]
	\centering
	\includegraphics[scale=0.30]{11}
	\caption{Creación del LV de la raíz.}
\end{figure}

Configuración para \textbf{/var.}

\begin{figure}[H]
	\centering
	\includegraphics[scale=0.30]{12}
	\caption{Creación del LV de la var.}
\end{figure}

Repetimos los mismos pasos para \textbf{/home} y esta sería la ventana resumen.

\begin{figure}[H]
	\centering
	\includegraphics[scale=0.30]{13}
	\caption{Creación del LV de home.}
\end{figure}

Confirmamos el particionado.

\begin{figure}[H]
	\centering
	\includegraphics[scale=0.30]{14}
	\caption{Confirmación del particionado.}
\end{figure}

Ahora configuramos el usuario-sudo y la contraseña.

\begin{figure}[H]
	\centering
	\includegraphics[scale=0.30]{15}
	\caption{Configuración del usuario.}
\end{figure}

Proceso de instalación del sistema base.

\begin{figure}[H]
	\centering
	\includegraphics[scale=0.30]{16}
	\caption{Instalación del sistema base.}
\end{figure}

Prueba final del sistema, con su particionado correcto, recordemos que usamos BTRFS como sistema de ficheros bajo LVM.

\begin{figure}[H]
	\centering
	\includegraphics[scale=0.30]{17}
	\caption{lsblk como demostración.}
\end{figure}



% \vspace{5mm}


% \begin{lstlisting}[style=mybash]
%     # Para una base de datos concreta
%     mysqldump --user=tiendabd --password=password --databases tiendabd --add-drop-database --add-drop-table [--replace] --host=127.0.0.1 --result-file=dump.sql
% \end{lstlisting}



%\begin{figure}[H]
%	\centering
%	\includegraphics[scale=0.40]{cuestion_1_1}
%	\caption{Se puede ver que al no haber un fallo grave, el sistema lo nota como que sigue funcionando pero en un estado degradado.}
%\end{figure}

%\newpage

%Se pueden hacer include en latex
%\input{plantilla_include.tex}


%-------Bibliografia-----------------------------

%\input{bibliografia.tex}

\end{document}
