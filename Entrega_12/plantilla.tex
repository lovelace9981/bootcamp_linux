\input{preambulo.tex}

%----------------------------------------------------------------------------------------
%	TÍTULO Y DATOS DEL ALUMNO
%----------------------------------------------------------------------------------------
\graphicspath{ {img/} }

\title{
\normalfont \normalsize
\includegraphics[width=6cm,height=6cm]{logo}\\
\textsc{\textbf{Bootcamp Especialidad GNU/Linux (2023)}} \\ [25pt] % Your university, school and/or department name(s)
\horrule{0.5pt} \\[0.4cm] % Thin top horizontal rule
\huge Lab 12 - Preparación de actualizaciones desatendidas \\ % The assignment title
\horrule{2pt} \\[0.5cm] % Thick bottom horizontal rule
}
%https://es.overleaf.com/learn/latex/Inserting_Images
%Ruta relativa de   imagenes

\author{Pedro Antonio Mayorgas Parejo} % Nombre y apellidos

\date{\normalsize\today} % Incluye la fecha actual

%----------------------------------------------------------------------------------------
% DOCUMENTO
%----------------------------------------------------------------------------------------

\begin{document}

\maketitle % Muestra el Título

\newpage %inserta un salto de página

\tableofcontents % para generar el índice de contenidos

\newpage

%----------------------------------------------------------------------------------------
%	Cuestión 1
%----------------------------------------------------------------------------------------

\section{Actualizaciones desatendidas en Red Hat}

Para poder realizar actualizaciones desatendidas instalamos el siguiente paquete.
\begin{lstlisting}[style=mybash]
sudo dnf install dnf-automatic
\end{lstlisting}

Luego para activar solamente las actualizaciones desatendidas tenemos que ir a la siguiente ruta \textbf{/etc/dnf/automatic.conf}, para poner lo siguiente.
% \vspace{5mm}

\begin{figure}[H]
	\centering
	\includegraphics[scale=0.30]{00}
	\caption{Instalación del paquete.}
\end{figure}

\begin{figure}[H]
	\centering
	\includegraphics[scale=0.30]{01}
	\caption{Permitiendo aplicar las actualizaciones desatendidas.}
\end{figure}

\begin{figure}[H]
	\centering
	\includegraphics[scale=0.30]{02}
	\caption{Cambiando el modo de notificación de actualizaciones desatendidas a mail.}
\end{figure}


\begin{figure}[H]
	\centering
	\includegraphics[scale=0.30]{03}
	\caption{Configurando el email, emisor y receptor.}
\end{figure}

% \begin{lstlisting}[style=mybash]
%     # Para una base de datos concreta
%     mysqldump --user=tiendabd --password=password --databases tiendabd --add-drop-database --add-drop-table [--replace] --host=127.0.0.1 --result-file=dump.sql
% \end{lstlisting}



%\begin{figure}[H]
%	\centering
%	\includegraphics[scale=0.30]{cuestion_1_1}
%	\caption{Se puede ver que al no haber un fallo grave, el sistema lo nota como que sigue funcionando pero en un estado degradado.}
%\end{figure}

%\newpage

%Se pueden hacer include en latex
%\input{plantilla_include.tex}


%-------Bibliografia-----------------------------

%\input{bibliografia.tex}

\end{document}
